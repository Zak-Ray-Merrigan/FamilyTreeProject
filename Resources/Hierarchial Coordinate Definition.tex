\documentclass{article}
\usepackage{amsmath, amssymb, amsthm}

\begin{document}

\section*{Hierarchical Coordinate of a Family Dynamic in a Family Tree by an Inherited Family Name}

Let \( G = (V, E) \) be a directed graph representing a family tree, where each vertex \( v \in V \) corresponds to a \textbf{single-sided Family Dynamic} encoded as a JSON object with the following fixed schema:
\begin{itemize}
    \item \textbf{Id}: A unique identifier for the vertex (Family Dynamic node).
    \item \textbf{Inherited Family Names} \( v_F \): A set of family names inherited by the member.
    \item \textbf{MemberId}: A reference to the person collection in the DocumentDB.
    \item \textbf{In-law Id} (optional): A reference to the in-law in the person collection.
    \item \textbf{FamilyDynamicId} (optional): A reference to the family dynamic record in the DocumentDB.
\end{itemize}

Each directed edge \( (v, u) \in E \) represents a \textbf{parent-child} relationship, indicating that \( u \) is a child of \( v \).

\subsection*{Subgraph Extraction by Family Name}

Given a specific inherited family name \( f \), define a subgraph \( G' = (V', E') \) such that:
\[
V' = \{ v \in V \mid f \in v_F \}
\]
and \( E' \) retains all edges from \( E \) that connect nodes within \( V' \). To ensure a tree structure, we introduce a synthetic root vertex \( r \), connected to each vertex in \( V' \) that has no parent in \( V' \). This yields a rooted tree:
\[
T = (V' \cup \{r\}, E' \cup E_r)
\]
where \( E_r \) is the set of edges from \( r \) to root-level Family Dynamics in the subgraph.

\subsection*{Hierarchical Coordinate Assignment}

Let \( w \in T \) be a target Family Dynamic. To compute its \textbf{hierarchical coordinate}:

\begin{enumerate}
    \item Find the unique path \( P = (r, v_1, v_2, \dots, w) \) from the root \( r \) to \( w \) in the tree \( T \).
    \item At each step along this path, enumerate the children of the current parent vertex \emph{from left to right}, assigning labels \( 1, 2, \dots \) in the order they appear among that parent's children.
    \item For each vertex \( v_i \in P \), let \( L(v_i) \) denote its \textbf{position among its siblings} (children of its parent).
    \item Collect the labels into an ordered list:
    \[
    \mathcal{H}(w) = [ L(v_1), L(v_2), \dots, L(w) ]
    \]
\end{enumerate}

This list uniquely encodes the \textbf{hierarchical position of \( w \)} within the tree rooted at \( r \), as filtered by the inherited family name \( f \).

\subsection*{Example}

Consider the following tree structure:

\[
\begin{array}{c}
(r) \\
\downarrow \\
(A) \\
/ \quad \backslash \\
(B) \quad (C) \\
| \quad \quad \backslash \\
(D) \quad \quad (E)
\end{array}
\]

\begin{itemize}
    \item \( A \) is the first child of \( r \): \( L(A) = 1 \)
    \item \( B \), \( C \) are children of \( A \): \( L(B) = 1 \), \( L(C) = 2 \)
    \item \( D \) is the first child of \( B \): \( L(D) = 1 \)
    \item \( E \) is the first child of \( C \): \( L(E) = 1 \)
\end{itemize}

Now consider the path \( r \to A \to C \to E \). The hierarchical coordinate of \( E \) is:
\[
\mathcal{H}(E) = [1, 2, 1]
\]

Note that although \( D \) and \( E \) are at the same depth, their sibling groups differ due to different parent vertices, and each is the first among its siblings.

\end{document}